\documentclass[conference]{IEEEtran}
\IEEEoverridecommandlockouts
\usepackage{amsmath,amssymb,amsfonts}
\usepackage{algorithm}
\usepackage{algorithmicx}
\usepackage{algpseudocode}
\algnewcommand\algorithmicforeach{\textbf{for each}}
\algdef{S}[FOR]{ForEach}[1]{\algorithmicforeach\ #1\ \algorithmicdo}
\algblock{Input}{EndInput}
\algnotext{EndInput}
\algblock{Output}{EndOutput}
\algnotext{EndOutput}
\newcommand{\Desc}[2]{\State \makebox[2em][l]{#1}#2}
\usepackage{graphicx}
\usepackage{caption}
\usepackage{subcaption}
\usepackage{textcomp}
\usepackage{xcolor}
\usepackage{hyperref}
\usepackage{xspace}
\usepackage{multirow}
\usepackage[binary-units]{siunitx}
\def\BibTeX{{\rm B\kern-.05em{\sc i\kern-.025em b}\kern-.08em
    T\kern-.1667em\lower.7ex\hbox{E}\kern-.125emX}}
\newcommand{\bigdata}{Big Data\xspace}
\newcommand{\bigbrain}{BigBrain\xspace}

\newcommand{\valerie}[1]{\color{blue}\textbf{Note from Valerie}:
      #1 \color{black}}
\newcommand{\tristan}[1]{\color{orange}\textbf{Note from Tristan}:
      #1 \color{black}}



\begin{document}

\title{Performance benefits of Intel Optane DC storage for the parallel processing of large neuroimaging data}

\author{\IEEEauthorblockN{Val\'erie Hayot-Sasson$^1$, Shawn T Brown$^2$ and 
    Tristan Glatard$^1$
  }\\
  \IEEEauthorblockA{
    $^1$Department of Computer Science and Software Engineering, Concordia University, Montreal, Canada\\
    $^2$Montreal Neurological Institute, McGill University, Montreal, Canada
  }
}
\maketitle

\begin{abstract}
    Open-access neuroimaging datasets have reached up to $\SI{}{\peta\byte}$ of data, and
    continue to grow. The ability to leverage the entirety of these datasets are
    limited to a restricted number of labs with both the capacity and infrastructure
    to process the data. Whereas BigData engines have significantly reduced
    application performance penalties with respect to data movement, their applied
    strategies are not necessarily practical within neuroimaging workflows where
    intermediary results may need to be materialized to shared storage for post-processing
    analysis. In this paper we evaluate the performance advantage brought by Intel
    Optane DC persistent memory storage for the processing of large neuroimaging 
    datasets using the two available configurations modes: Memory Mode and App Direct Mode.
    We employ a synthetic algorithm on the $\SI{76}{\gibi\byte}$ \bigbrain
    and a real neuroimaging application on the Consortium for Reliability and Reproducibility (CoRR)
    dataset using 25 and 96 parallel
    processes in both cases. Our results show that performance of applications executing on
    Optane is superior to that of other storage devices, with the exception of DRAM. This is
    the case in both Memory and App Direct Mode and irrespective of the amount of data and 
    parallelism. Furthermore, Optane in App Direct Mode is believed to benefit from the use
    of memory as a cache for writing in cases where output data is significantly smaller than
    available memory. We believe the use of Optane will be beneficial to both neuroimaging applications
    running on HPC or visualization of large, high-resolution images.
\end{abstract}

\section{Introduction}
Neuroimaging open-data initiatives have led to extensively large repositories of
publicly available data. Such initiatives include the \bigbrain~\cite{BigBrain}, 
a one-of-a-kind $\SI{603}{\gibi\byte}$
histological image of a 65 year-old healthy human brain
at 20$\mu$m resolution; the UK Biobank~\cite{ukbiobank}, a repository expected to
contain approximately $\SI{0.2}{\peta\byte}$ of data (including various magnetic
resonance (MR) imaging modalities) from 500,000 individuals living in the UK;
the Human Connectome Project~\cite{HCP}, a repository containing MR scans from
1,200 healthy adults, which is expected to exceed $\SI{2}{\peta\byte}$ in size; 
and the Consortium for Reliability and Reproducibility (CoRR)~\cite{corr}, an
initiative which aggregates MR data from various centres around the world, 
of which 32 are currently available and make up about $\SI{937}{\giga\byte}$ of
data in total.

Due to storage limitations, only subsets of such neuroimaging repositories 
can be processed in a typical research laboratory. Moreover, as these datasets 
are extremely large and are only increasing in size, they are typically stored 
in higher-capacity, slower storage devices, such as hard disk drives, or external
parallel file systems, such as Lustre. In such conditions, large-scale 
studies in neuroimaging remain limited to labs with access to adequate 
infrastructure.

Furthermore, intermediary data is often required for post-processing analysis. This limits
any performance benefits that can arise from volatile in-memory computing as intermediary
results need to be materialized onto persistent storage. As neuroimaging datasets continue
to increase in size, the movement of data will significantly increase the processing time.

To mitigate the effects of data writes, kernel-based
strategies,such as the writeback cache, have been developed. The writeback cache
allows processes to use the memory as a cache for writing. The cache size,
however,
is configurable, but nevertheless limited. When writes approach the cache's capacity,
processes performing writes start to be throttled. After capacity is reached,
processes can no longer use the cache until all written data on cache is flushed
to the appropriate storage device.

Whereas Operating Systems and popular Big Data engines, such as MapReduce~\cite{mapreduce} and Apache Spark~\cite{spark}, have
incorporated software solutions to limit data transfers (e.g. writeback cache, in-memory computing,
data locality, and lazy evaluation), hardware has also adapted to the growing datasets.
One such improvement is the concept of placing flash storage directly on the Dual
In-line Memory Module (DIMM), thereby reducing the latency of accessing data on 
flash storage devices. While the latency of these devices is improved, the bandwidth
remains the same. However, as noted by \cite{nvdimms}, a severe performance degradation
can be experienced by having memory traffic and I/O placed on the same bus.


Intel Optane DC Persistent Memory storage~\cite{optanebrief} is a recently introduced high-performance
storage technology that also sits on the DIMMs to reduce latency to the device.
Unlike DRAM DIMMs, it can accommodate twice the volume ($\SI{512}{\giga\byte}$) of
storage enabling it to be a more cost effective alternative to DRAM storage.
There are two principle configuration modes, namely Memory Mode and App Direct Mode,
that enable the storage to either be accessed as an extension of available memory 
or as a non-volatile memory storage device.


In this paper, we aim to:
\begin{itemize}
        \item Quantify the added value of Intel Optane DC Persistent Memory on 
            processing large neuroimaging data using representative pipelines; and
        \item Determine when a given Intel Optane storage configuration (Memory 
            and App Direct Mode) is preferable.
\end{itemize}

\section{Materials and Methods}
The application pipelines, benchmarks, performance data, and analysis scripts used 
to implement the methods described hereafter are all available at 
\url{https://github.com/valhayot/paper-memory-storage} for 
further inspection and reproducibility.

\subsection{Infrastructure}

The server used for the experiments consisted of 12 \SI{64}{\gibi\byte} DRAM devices,
resulting in a total of \SI{768}{\gibi\byte} of DRAM, and 12 \SI{256}{\giga\byte} Intel Optane DC Memory
devices, resulting in total of \SI{3}{\tebi\byte} of Intel Optane DC Memory. The theoretical maximum bandwidth
of DRAM is estimated to be around \SI{275}{\gibi\byte}~/s whereas for Optane, it is
estimated to be around \SI{250}{\gibi\byte}~/s.
Other storage devices included a \SI{240}{\gibi\byte} Micron SATA SSD, of which
only \SI{149}{\gibi\byte} where available for the experiments, and a
\SI{720}{\tebi\byte} shared Dell EMC Isilon cluster. The Isilon cluster is an nsf4 mounted
cluster (10Gb network) consisting of 5 nodes of 36 hard disk drives (HDD). The local disk
was the only storage device set up as a writeback device. Both Isilon and Optane in
App Direct mode were configured as write-through devices, meaning they did not leverage
DRAM as a cache for writes. This was achieved in Optane by configuring the xfs filesystem with
DAX. All storage devices were benchmarked using the following \href{https://github.com/ValHayot/paper-memory-storage/scripts/bench_disks.sh}{script}.
The result of the benchmarks can be seen in Table \ref{table:bandwidths}.

For processing,
2 Intel(R) Xeon(R) Platinum 8260M CPU @ 2.40GHz where installed, enabling use of up to 96 threads.
The server was running Red Hat Enterprise Linux 7.6 release (Maipo). 


\begin{table}
\begin{center}
 \begin{tabular}{ |c|c|c| } 
     \cline{2-3}
     \multicolumn{1}{c|}{} & \multicolumn{2}{c|}{Bandwidth (MB/s)} \\\hline
  Device & Read & Write \\
 \hline
 DRAM & 5304.3 & 3338.2 \\  
 Optane & 3379.2 & 2396.2 \\   
 Local Disk & 518.6 & 240.4 \\
 Isilon & 117.0 & 111.8 \\
 \hline
\end{tabular}\caption{Perceived read and write bandwidth of each storage device}\label{table:bandwidths}
\end{center}
\end{table}



\subsection{Storage configuration}

\subsubsection{Memory Mode}

Memory Mode leverages Intel Optane DC persistent memory (Optane) to extend the system's available
memory. In this mode, Optane uses DRAM as a cache and is accessible as
volatile addressable main memory. By extending main memory, Memory Mode enables the
fast access of large volumes of data.

While enabling the access to larger amount of memory than typically accessible otherwise,
it is anticipated that Memory Mode will be slower than AD mode for all memory accessed that
are not cached in DRAM. We evaluated all available devices (Optane through tmpfs, local disk and Isilon)
in Memory Mode.

\subsubsection{App Direct Mode}

App Direct Mode enables Optane to be accessed as a high-performance storage device.
Unlike Memory Mode, the OS is able to differentiate between DRAM and Optane,
treating them as two distinct entities. As Optane is located on
the DIMMs, it exhibits lower latencies than traditional SSDs, while being able
to be accessed like an SSD. Optane does not use the DRAM as cache in App Direct Mode
if configured with DAX.

Similarly to Memory Mode, our experiments evaluated all available filesystems in App Direct
Mode. In other words, we evaluated DRAM performance through tmpfs, Optane performance through
a \SI{1.5}{\tebi\byte} Intel Optane DC Persistent Memory storage device and local disk
and Isilon using the same storage mounts available in Memory Mode.


\subsection{Performance model}

We characterize the performance of our experiments using the following model:

\begin{equation}
    M \geq \frac{D}{R} + \frac{D}{W} \label{eq:makespan}
\end{equation}

where,
\begin{itemize}
        \item $M$ is the application makespan
        \item $D$ is the total amount of data processed by the application
        \item $R$ is the device read bandwidth
        \item $W$ is the device write bandwidth
\end{itemize}

For applications which have negligible CPU time, as is the case with our experiments, 
it is expected that the I/O duration can estimate the total makespan. However, there
are certain instances where the makespan may be below the I/O duration. This is expected
to occur with storage devices that scale well with increased parallelism, as should be the
case for DRAM, Optane and Isilon, which all consist of multiple devices. Should the
scalable storage devices predict the makespan accurately, it is believed that other overheads
would be at play making the device unfavourable for parallel processing.

\subsection{Applications}
\subsubsection{ \bigbrain Incrementation}

\begin{algorithm}\caption{Incrementation}\label{alg:incrementation}
    \begin{algorithmic}[1]
    \Input
        \Desc{$x$}{a sleep delay in seconds}
        \Desc{$n$}{a number of iterations}
        \Desc{$C$}{a set of image chunks}
        \Desc{$fs$}{filesystem to write to (tmpfs, Optane, local disk, Isilon)}
    \EndInput
    \ForEach{$chunk \in C$}
        \State read $chunk$ from $fs$
        \State $chunk\gets chunk+1$
        \State save $chunk$ to $fs$
    \EndFor
    \end{algorithmic}
\end{algorithm}  

Due to the size and uniqueness of the \bigbrain, standardized processing pipelines
have yet to be developed. In order to quantify the effects of storage devices
on such a dataset, we have implemented a naive synthetic application that takes 
the image, split into blocks, and increments all the voxels within each block, in parallel (Algorithm~\ref{alg:incrementation}).
This application
enables us to read and write to different storage devices and ensure that written
data could not have been previously cache
in-memory, by ensuring that read and written data are not the same. This application
was parallelized us in two different ways: using both GNU Parallel and Apache Spark 2.4.3 (PySpark). 
All code was implemented in Python 3.6.

There are some notable differences between the GNU Parallel and Apache Spark implementations.
For instance, in the GNU Parallel implementation, all operations (i.e. reading, incrementing and writing)
are done within the same task. The same task is applied to all the \bigbrain blocks and GNU Parallel is 
provided with all the different command-line calls, processing a subset of them at a time,
depending on the level of parallelism provided. Reading in the GNU Parallel implementation 
is achieved using the popular neuroimaging I/O: Nibabel. When provided with a filename, as is the case
for the GNU Parallel implementation, Nibabel will simply load the header in memory, and memmap the data
using Numpy. This step will be referred to as ``load header''. It is only when the data is actually required
(i.e. during incrementation), that the data will be loaded into memory.

The Apache Spark implementation differs from that of GNU Parallel in that scheduling decisions
were entirely differed to Spark. That is, rather than simply parallelizing the filenames, to be
loaded in a similar fashion to that of GNU Parallel, we chose to load the using Spark's built-in
\texttt{BinaryFiles}, which loads whole files, in binary format, into Spark partitions. As the
data would have been pre-loaded by Spark, ``loading header'' only measures the time to
convert the binary images into Nibabel objects, and ``incrementation'' consists purely
of the time it took to increment the data. Therefore, only the write time would be
comparable between the two implementations, as it is achieved using the same method.
Furthermore, the Spark implementation also differs in that read, increment and write
were separated into three map tasks, which can enable shuffling between the tasks should 
it be determined as necessary by the Spark scheduler.

Time is measured within the application using Python's \texttt{time} module 
called before and after any read and writes.

We have executed this pipeline on both the $\SI{75}{\gibi\byte}$ 40~$\mu$m 
\bigbrain split into 125 $\SI{614}{\mebi\byte}$ blocks and the $\SI{603}{\gibi\byte}$
\bigbrain at 20~$\mu$m split into 1000 $\SI{617}{\mebi\byte}$ blocks.

For the 40~$\mu$m \bigbrain, we have executed the application using GNU Parallel for parallelization
using 25 and 96 processes on the 125 40$\mu$m \bigbrain blocks. Data was read and written to either
DRAM (App Direct only), Optane, local disk and Isilon. 

For the 20~$\mu$m \bigbrain, we used the same application, however in this case, it was executed
using both Apache Spark and GNU Parallel. The configuration was also generally the same, having 
experiments using both 25 and 96 cpus. Repeating the experiments using a larger dataset would enable
us to determine the effects when Optane would have to be partially relied upon due to insufficient DRAM
space. Moreover, with such large datasets, it is likely that Big Data frameworks would be used rather than
more traditional parallelism frameworks. Using Spark, we can evaluate how Big Data Frameworks perform on
different storage devices.

Since storage was limited on DRAM and local disk and was not large enough to process the entire
20~$\mu$m \bigbrain, these devices were omitted in the processed of this dataset. 

\subsubsection{BIDS App Example}

The BIDS App Example is a template example for creating a Brain Imaging Data Structure (BIDS)
compliant application. It runs a standard neuroimaging brain extraction application (FSL BET) 
on all the anatomical (T1W) images of dataset containing numerous subjects. This step is 
referred to as Participant analysis within the BIDS App Example. An optional
step of the BIDS App example, referred to as Group analysis, computes the average brain
mask size of the entire dataset.

For our experiments, we used the entire CoRR dataset, made available on \href{}{DataLad} and
applied Participant analysis to it. The BIDS App Example was executed using a Singularity container stored on
Isilon. As in the BigBrain Incrementation, the experiment parallelized 
using GNU Parallel with 25 and 96 processes. The conditions were executed in both App Direct and Memory Mode
using Isilon, local disk and Optane as storage. Each experiment was repeated 3x.

Time in this application was obtained using Linux's \texttt{time(1)} application.
\texttt{Real + Sys} was used to measure CPU time,
whereas \texttt{User - (Real + Sys)} was used to measure
I/O time.
\section{Results}

\subsection{40~$\mu$m \bigbrain incrementation}
\subsubsection{25 Processes}
\begin{figure}
    \includegraphics[width=\columnwidth]{./experiments/bigbrain-incrementation/figures/makespan-real-40bb_25cpus.pdf}
    \captionsetup{belowskip=-10pt}
    \caption{Makespan of the \bigbrain Incrementation application using 25 
             processes on the 125 blocks on the 40~$\mu$m image given different
             storage configurations. Anticipated read and write duration was 
             measured by applying the perceived bandwidths Table~\ref{table:bandwidths}
             to Eq.~\ref{eq:makespan}. 3 
             repetitions were performed. *Local SSD did not complete the
    writing of the last few blocks due to storage limitations.}\label{fig:makespan-25cpus}
\end{figure}
\begin{figure}
    \includegraphics[width=\columnwidth]{./experiments/bigbrain-incrementation/figures/stacked-real-40bb_25cpus.pdf}
    \captionsetup{belowskip=0pt}
    \caption{Total read/increment/write breakdown of \bigbrain Incrementation using 25 cpus. 3 repetitions were performed. *Local SSD did not
             complete the writing of the last few blocks due to storage limitations}\label{fig:stacked-25cpus}
\end{figure}
\begin{figure*}
    \begin{subfigure}{\columnwidth}
        \centering
    \includegraphics[width=\columnwidth]{./experiments/bigbrain-incrementation/figures/gantt-1572380543-tmpfsAD_1it_25cpus_40bb-1.pdf}
    \caption{DRAM}
\end{subfigure}
\begin{subfigure}{\columnwidth}
        \centering
    \includegraphics[width=\columnwidth]{./experiments/bigbrain-incrementation/figures/gantt-1572382482-optaneAD_1it_25cpus_40bb-1.pdf}
    \caption{Optane}
\end{subfigure}
\begin{subfigure}{\columnwidth}
        \centering
    \includegraphics[width=\columnwidth]{./experiments/bigbrain-incrementation/figures/gantt-1572382794-localAD_1it_25cpus_40bb-1.pdf}
    \caption{local SSD*}
\end{subfigure}
\begin{subfigure}{\columnwidth}
        \centering
    \includegraphics[width=\columnwidth]{./experiments/bigbrain-incrementation/figures/gantt-1572388118-isilonAD_1it_25cpus_40bb-2.pdf}
    \caption{Isilon}\label{fig:gantt25isilon}
\end{subfigure}
    \captionsetup{belowskip=-10pt}
\caption{Gantt charts for each storage device (App Direct Mode) processing 125 blocks of the 40$\mu$m BigBrain using 25 processes. *Some local SSD writes did not complete due to storage limitations.}\label{fig:gantt25}
\end{figure*}

As can be seen in Figure~\ref{fig:makespan-25cpus}, DRAM (App Direct Mode) is the 
most efficient, with a makespan of 7.6s. Optane in Memory mode is a close second
with a makespan of approximately 10s. Optane in App Direct mode is approximately 
12x slower than Optane in Memory Mode. Isilon in both Memory Mode and App Direct 
Mode is more than 120x slower than Optane in Memory Mode. Other than for Optane, 
there is no significant difference between Memory Mode and App Direct executions.

When comparing the makespan to the estimated read and write duration, local SSD
in both modes and Optane in App Direct Mode have longer makespans than the anticipated.
Optane was approximately 2.2x slower than expected, whereas local SSD was approximately
1.4x slower than expected, in both modes. For those where the I/O time was overestimated,
DRAM was almost 5x faster in reality compared to what was estimated. Optane in Memory Mode
had a smaller difference, with it being 3.8x faster than expected. Isilon in App Direct
mode had a larger difference than Memory mode, with it being 1.1x faster than expected.


The total task duration breakdowns for each device (Figure~\ref{fig:stacked-25cpus})
shows that, as expected, I/O times vary greatly between devices, with DRAM having the
best total read and write speeds. Optane in Memory Mode is very close to DRAM in speed,
whereas Optane in App Direct mode 7x slower in terms of reading and 23x slower in terms
of writing. Local SSD, on the other hand, exhibits similar write times as DRAM, but is
approximately 96x that DRAM with respect to reads. Header loading for local SSD is also 90x longer
than that of DRAM, however, remains negligible compared to other tasks. Isilon, on the 
other hand, has non negligible header load times, with it being more than 5000 times
slower than that of DRAM. Interestingly enough, the read/increment times on Isilon are
slower than that of the local SSD, although Isilon makes up for its speedier reads with 
writes that are around 184x slower than DRAM.


The App Direct mode Gantt charts (Figure~\ref{fig:gantt25}) also reflect what is observed in Figure~\ref{fig:stacked-25cpus}.
DRAM is significantly faster than all other storage, leading it to barely appear within the Gantt chart.
DRAM was measured to have an average of 17.2 parallel tasks throughout the execution. Optane takes
significantly longer than DRAM and appears to spend a significant amount of time writing. The average
parallelism measured for Optane was 24. Unlike Optane, local SDD spends the vast majority of its time
reading, while spending a negligible amount of time writing. Like Optane, it averages 24 parallel tasks
throughout its execution. Unlike the other three storages, Isilon shows spacing between the loading of the header,
read and increment and write. Isilon was measured to have an average parallelism of 14. Overall, each storage device
execution displayed that some tasks took longer than others.

\subsubsection{96 Processes}
\begin{figure}
    \includegraphics[width=\columnwidth]{./experiments/bigbrain-incrementation/figures/makespan-real-40bb_96cpus.pdf}
    \captionsetup{belowskip=-10pt}
    \caption{Makespan of the \bigbrain Incrementation application using 96 
             processes on the 125 blocks on the 40~$\mu$m image given different
             storage configurations. Anticipated read and write duration was 
             measured by applying the perceived bandwidths Table~\ref{table:bandwidths}
             to Eq.~\ref{eq:makespan}. 3 
             repetitions were performed. *Local SSD did not complete the
    writing of the last few blocks due to storage limitations.}\label{fig:makespan-96cpus}
\end{figure}
\begin{figure}
    \includegraphics[width=\columnwidth]{./experiments/bigbrain-incrementation/figures/stacked-real-40bb_96cpus.pdf}
    \captionsetup{belowskip=-10pt}
    \caption{Total read/increment/write breakdown of \bigbrain Incrementation using 25 cpus. 3 repetitions were performed. *Local SSD did not
             complete the writing of the last few blocks due to storage limitations}\label{fig:stacked-96cpus}
\end{figure}

\begin{figure*}
    \begin{subfigure}{\columnwidth}
        \centering
    \includegraphics[width=\columnwidth]{./experiments/bigbrain-incrementation/figures/gantt-1572306855-tmpfs_1it_96cpus_40bb-2.pdf}
    \caption{Memory Mode}
\end{subfigure}
\begin{subfigure}{\columnwidth}
        \centering
    \includegraphics[width=\columnwidth]{./experiments/bigbrain-incrementation/figures/gantt-1572379747-optaneAD_1it_96cpus_40bb-1.pdf}
    \caption{App Direct Mode}
\end{subfigure}
    \captionsetup{belowskip=-10pt}
\caption{Gantt charts for Optane processing 125 blocks of the 40$\mu$m BigBrain using 96 processes}\label{fig:gantt96}
\end{figure*}

Despite having increased parallelism by a factor of 3.84, we see no visible reduction
in makespan (Figure~\ref{fig:makespan-96cpus}). At 96 cpus, DRAM and Optane in Memory Mode are the only storage
devices that did better than the anticipated I/O, with a makespan duration of
5.7 and 9s, respectively (total read and write estimate for both was measured to be around 39s). While
Optane in App Direct Mode and local SSD both performed worse than the anticipated I/O estimates, Isilon
performed as expected. Variance was found to be high on the local SSD.

The total task duration breakdowns (Figure~\ref{fig:stacked-96cpus}) show that total task durations
have nearly quadrupled. Whereas most devices decreased in performance in both reading and writing,
Isilon only appeared to display a performance decrease with respect to writes. Furthermore, while both
read and writes decreased for Optane in App Direct Mode, read duration was nearly 6x slower with more threads,
while write duration was only about 2x slower. For local SSD and Isilon, there does not appear to 
be any significant difference between App Direct and Memory mode.

When analyzing the average parallelism during the workload, only Optane in App Direct mode
came close to 96 parallel tasks, having an average parallelism of 88 tasks. Optane in Memory mode,
DRAM and local SSD all averaged between 61-66 tasks. Isilon performed the worst with 38 tasks in Memory 
mode and 45 tasks in App Direct mode.

A further look at the Gantt charts between Optane in Memory Mode and App Direct Mode (Figure~\ref{fig:gantt96})
show that both read and write time of tasks are worse in App Direct mode as compare with Memory mode.
While some reads tasks in App Direct mode appear to be of similar duration to the average read task time
of Memory mode, the vast majority of read tasks are significantly slower. Futhermore, not write tasks in App
Direct mode is capable of reaching the performance of equivalent tasks in Memory mode.

\subsection{20~$\mu$ \bigbrain incrementation}
\subsubsection{25 Processes}

The anticipated I/O estimates did not correctly predict the makespan for 25 parallel
processes (Figure~\ref{fig:20mksp25}). Both Optane in Memory Mode and Isilon appear 
to be under the estimates: Optane in Memory Mode is approximately twice as fast as the
estimates whereas Isilon is around 1.1x faster. Optane was 1.8x slower than anticipated.
Once again, Memory and App Direct Mode appear to only affect performance for Optane.

Total task duration breakdowns (Figure~\ref{fig:20total25}) show that for Optane in
App Direct Mode spends significantly more time writing than its Memory Mode counterpart.
In fact, it spends almost 12x more time writing. Isilon also appears to vary slightly
between Memory and App Direct Mode, spending around 1.4x more time reading and incrementing
in App Direct Mode. While the read and write duration appeared to be more-or-less balanced in
Isilon (as was anticipated), Optane in App Direct Mode spent significantly more time writing, whereas the 
inverse was apparent in Memory Mode

Unlike the GNU Parallel executions, the Spark executions appear to be in better accordance with
the I/O estimates (Figure~\ref{fig:20mkspspark25}. Only Optane in App Direct Mode appears
to have significantly exceeded the estimates, by a factor of 2. Total write duration all appear
to be longer in the Spark implementation than with GNU Parallel (Figure~\ref{fig:20totalspark:25}).
Spark's writes are 1.4-1.6x longer with Optane and 1.1x longer with Isilon. Due to differences in
how data is loaded in Spark, it is not possible to comment on how read times are affected. However, it
is possible to note that data conversion (binary string to Numpy array) takes almost twice as 
much time in Optane than it does with Isilon. Furthermore, the act of incrementation can take
up to twice as much time with Isilon.

\begin{figure*}
    \begin{subfigure}{\columnwidth}
        \centering
    \includegraphics[width=\columnwidth]{./experiments/bigbrain-incrementation/figures/makespan-real-20bb_25cpus.pdf}
    \caption{Makespan}\label{fig:20mksp25}
\end{subfigure}
\begin{subfigure}{\columnwidth}
        \centering
    \includegraphics[width=\columnwidth]{./experiments/bigbrain-incrementation/figures/stacked-real-20bb_25cpus.pdf}
    \caption{Total read/increment/write breakdown}\label{fig:20total25}
\end{subfigure}
\captionsetup{belowskip=-10pt}
\caption{GNU Parallel incrementation application processing the 20~$\mu$m BigBrain using
25 processes. 3 repetitions were performed.}\label{fig:2025}
\end{figure*}

\begin{figure*}
    \begin{subfigure}{\columnwidth}
        \centering
    \includegraphics[width=\columnwidth]{./experiments/bigbrain-incrementation/figures/makespan-real-spark_20bb_25cpus.pdf}
    \caption{Makespan}\label{fig:20mkspspark25}
\end{subfigure}
\begin{subfigure}{\columnwidth}
        \centering
    \includegraphics[width=\columnwidth]{./experiments/bigbrain-incrementation/figures/stacked-real-spark-20bb_25cpus.pdf}
    \caption{Total convert/increment/write breadown}\label{fig:20totalspark:25}
\end{subfigure}
\caption{Spark incrementation application processing the 20~$\mu$m BigBrain using 25 processes. 3 repetitions wereperformed}\label{fig:20spark25}
\end{figure*}

\subsubsection{96 Processes}

A very slight performance improvement (1-1.3x faster) can be witnessed in Optane. In 
contrast, Isilon 
experienced a slight performance decrease (1.1x slower). Whereas Isilon makespan appears to match
estimated I/O times, Optane once again has a superior Memory Mode makespan and an inferior App Direct
Mode makespan (Figure~\ref{fig:20mksp96}). As previously observed with 25 processes,
the bulk of the processing time is taken up by writing for Optane in App Direct Mode and 
Isilon in both modes (Figure~\ref{fig:20total96}). Reading takes more time in the case of Optane in Memory mode. 


A notable difference in the Spark execution is that Optane in Memory and App Direct Mode
exhibit the same performance (Figure~\ref{20mkspnspark96}). Both of which also performed
significantly worse than the estimated I/O duration. Performance on Isilon, however, did
not really differ. The total task breakdown shows that, as expected, write times are more
significant on Isilon than on Optane and Optane in App Direct Mode spends more time writing
than in Memory Mode (Figure~\ref{20fig:20totalspark96}). When comparing GNU Parallel and 
Spark implementations, Spark spent nearly half as much time writing as GNU Parallel. However,
it was found that Optane spent more time converting the data to Nibabel than GNU Parallel spent
in loading the header.

\begin{figure*}
    \begin{subfigure}{\columnwidth}
        \centering
    \includegraphics[width=\columnwidth]{./experiments/bigbrain-incrementation/figures/makespan-real-20bb_96cpus.pdf}
    \caption{Makespan}\label{fig:20mksp96}
\end{subfigure}
\begin{subfigure}{\columnwidth}
        \centering
    \includegraphics[width=\columnwidth]{./experiments/bigbrain-incrementation/figures/stacked-real-20bb_96cpus.pdf}
    \caption{Total read/increment/write breakdown}\label{fig:20total96}
\end{subfigure}
\caption{GNU Parallel incrementation application processing the 20~$\mu$m BigBrain using
96 processes. 3 repetitions were performed.}\label{fig:2096}
\end{figure*}

\begin{figure*}
    \begin{subfigure}{\columnwidth}
        \centering
    \includegraphics[width=\columnwidth]{./experiments/bigbrain-incrementation/figures/makespan-real-spark-20bb_96cpus.pdf}
    \caption{Makespan}\label{fig:20mkspspark96}
\end{subfigure}
\begin{subfigure}{\columnwidth}
        \centering
    \includegraphics[width=\columnwidth]{./experiments/bigbrain-incrementation/figures/stacked-real-spark-20bb_96cpus.pdf}
    \caption{Total convert/increment/write breakdown}\label{fig:20totalspark96}
\end{subfigure}
\caption{Spark incrementation application processing the 20~$\mu$m BigBrain using 96 processes. 3 repetitions were performed}\label{fig:20stackedp96}
\end{figure*}
\subsection{BIDS App Example}
\begin{figure*}
    \begin{subfigure}{\columnwidth}
        \centering
    \includegraphics[width=\columnwidth]{./experiments/bids-app-example/25cores.pdf}
    \caption{25 processes}\label{fig:bm25}
\end{subfigure}
    \begin{subfigure}{\columnwidth}
        \centering
    \includegraphics[width=\columnwidth]{./experiments/bids-app-example/96cores.pdf}
    \caption{96 processes}\label{fig:bm96}
\end{subfigure}
\caption{Makespan of BIDS App Example 25 and 96 processes on all storage devices. 3 repetitions were performed}
\end{figure*}

\begin{figure*}
    \begin{subfigure}{\columnwidth}
        \centering
    \includegraphics[width=\columnwidth]{./experiments/bids-app-example/25cores-sum.pdf}
    \caption{25 processes}\label{fig:bb25}
\end{subfigure}
    \begin{subfigure}{\columnwidth}
        \centering
    \includegraphics[width=\columnwidth]{./experiments/bids-app-example/96cores-sum.pdf}
    \caption{96 processes}\label{fig:bb96}
\end{subfigure}
\caption{I/O and CPU breakdown of BIDS App Example 25 and 96 processes on all storage devices. 3 repetitions were performed}\label{fig:bbd}
\end{figure*}
Unlike the BigBrain incrementation, the makespan of the BIDS App Example
executing on Optane is longer than local disk (Figure~\ref{fig:bm25}). However, similarly
to the BigBrain Incrementation, Optane in Memory Mode performs better than the other storage devices
while Isilon has the longest makespan. The same pattern can be found with 96 process (Figure~\ref{fig:bm96}). When comparing
the executions of 25 and 96 processes, the performance improved by a similar factor on all
storage devices.

The CPU and I/O breakdowns (Figure~\ref{fig:bbd}) show that there is a significant increase
in the time spent on I/O at 96 processes when compared to 25 processes. Furthermore, there appears
to also be an increase in the amount of time spent on CPU processing. In both cases,
tmpfs spends the least amount of time on I/O and CPU, followed by local disk, Optane, and finally,
Isilon.
\section{Discussion}
\subsection{Memory vs App Direct Mode}

In general, the selection of storage mode did not affect overall performance. This can be
due to several reasons. For instance, the amount of available DRAM was abundant and exceeded 
all our dataset sizes. While total memory used by the application may have exceeded DRAM, as is the
case with the 20~$\mu$ \bigbrain, the amount of memory required by the application at any 
given moment did not exceed the amount of available DRAM. Although GNU Parallel would have
been able to load all the blocks in memory, had we increased the block size, it would have been unlikely
that total available DRAM would have been exceeded, though it would have occurred in Memory Mode
for the devices that leverage memory as a write cache. In our case, only the local SSD was set up to do so.

Optane was the only device affected by the choice of storage mode. In all cases, with the 
exception of the Spark incrementation of the 20~$\mu$ \bigbrain, Memory Mode was superior to 
App Direct Mode. This is a result of Optane using DRAM as cache in Memory Mode, whereas Optane
was configured as a writethrough device in App Direct Mode. It is believed that had Optane been
configured to use DRAM as a writeback cache in App Direct Mode, we  would have observed a similar
performance between the two. However, for the 40~$\mu$ \bigbrain executions, read time on Optane in
App Direct Mode appeared to be slower than Memory Mode, potentially indicating that the Optane was able
to keep the input data cached in DRAM whereas in App Direct Mode, the data had to be loaded from Optane.
Therefore, it is expected the Optane in App Direct mode may always have longer read times than Memory Mode
if the data is already cached in DRAM in Memory Mode.

The Apache Spark and GNU Parallel implementations differed vastly in makespan 
for Optane in Memory Mode, particularly at 96 processes. After taking a closer look at
the captured Spark metrics, it was for that what really varied between the Spark runs on
the different storage was garbage collection, with garbage collection taking longer on Optane 
than on Isilon. Furthermore, garbage collection was longer in Optane Memory Mode than in 
Optane App Direct Mode. It is suspected that, in these cases, task duration was so short
that garbage collection could not keep up. Moreover, Optane in Memory Mode may have longer
garbage collection if old anonymous data cannot be located in the DRAM cache due to overall memory 
requirements, therefore requiring garbage collection to occur on slower memory.

\subsection{Effect of Page Cache}

The use of Optane in Memory Mode would enable more dirty data to be written to available memory
rather than directly to slower storage devices. In our 40~$\mu$ \bigbrain experiments, it was found
that read times were less important in Isilon than the local SSD (Figures~\ref{fig:stacked-25cpus}
and~\ref{fig:stacked-96cpus}). Should Isilon have been given the opportunity to benefit from a 
writeback cache like the SSD, it is believed that performance on Isilon could have been superior 
to that of the SSD. Having Optane as a node-local burst buffer to shared network storage 
may prove to be very beneficial in the case of Big Data neuroimaging pipelines running on
High Performance Computing clusters.

For App Direct Mode applications (although also applicable to Memory Mode), the 
choice of using the writeback cache depends on total amount of available memory, application
memory usage and I/O. Should the application be able to use the page cache without ever being
I/O-throttled, using a write-back cache will significantly improve performance. This can 
be seen in Figure~\ref{fig:stacked-25cpus}, for instance, where the local SSD is writing
entirely to memory, reducing significantly the total write time. However, should write operations
become throttled, this may significantly slow down the write task duration. In such cases,
it may be preferable to set them up as write-through devices. As Big Data neuroimaging applications
tend to generate temporary data files that are larger than input size, it is likely that maintaining the
devices as write-through is preferable. Furthermore, despite that Optane in App Direct Mode does
not leverage the writeback cache like the local SSD, is it still found to be superior in performance.
Therefore, the while the performance of Optane may not be enhanced by DRAM, it is still superior to that of
SSDs using the writeback cache and scalable network storage.

\subsection{Device scalability}
The number of storage devices attached to a particular filesystem mountpoint varied between
each storage device. For instance, DRAM consisted of 12 devices, whereas although there were 
12 Optane devices, only 6 were accessible by the mountpoint selected in App Direct Mode. Furthermore,
local SSD only consisted of a single device, whereas the Optane server was made up of 180 SSDs. The differences
in number of storage devices would account for the attainable amount of parallel writes. For instance, 
Isilon would have been the most scalable, followed by DRAM, Optane and the local SSD, which would
only be able to process I/O sequentially. This explains why the anticipated I/O time, is sometimes greater
than the real makespan of the pipeline.

For instance, taking into consideration the total number of DRAM devices, the expected makespan of the
40~$\mu$ \bigbrain is estimated to take around 3s, whereas sequentially it was measured to take
around 37.5s. At 25 processes, we found the DRAM makespan to be around 7s, which is much closer to the 3s
estimate than the 37.5s. Reasons for why it took longer that estimated could be due to application
overheads, or even that some DRAM devices were occupied, and therefore, the maximum amount of parallelism 
was not achievable. 

Despite the fact that only sequential I/O was considered for the anticipated I/O times, 
Optane in App Direct Mode always took more time than its sequential estimates. The cause
of the longer-than-expected makespan seems to be due to write times. While there may be application overheads,
they should, in theory, affect Optane in App Direct Mode as much DRAM. While it is unknown
why Optane writes are performing so poorly, it could be a result of filesystem configuration or 
inaccurate benchmarking of the disk. Otherwise, Optane does not appear to scale very well with respect to
writes. Using the storage benchmarks found in Table~\ref{table:bandwidths}, it is found that with an equivalent
number of storage devices, processing the 40~$\mu$ \bigbrain using 25 processes would be faster using
local disk when compared to the real times obtained on Optane. As there are overheads with both Optane and local
SSD, it is uncertain that it would, in fact, be the case.

While Isilon is, in theory, the most scalable device, its performance is expected. The Isilon
server is made up entirely of Hard Disks Drives, which are the slowest type of device compared to
the others, as seen in Table~\ref{table:bandwidths}. While Isilon displays a sufficient amount of
scalability to read and write the data in a few seconds, it is a network-backed device. As a result,
performance of this device is limited by the network speed of 10Gbps. Consequently, Isilon performs at around
the same rate as sequential writes.

Another interesting aspect of Isilon is the spacing that was observed between the read and writes
(Figure~\ref{fig:gantt25isilon}). What occurred between the read/increment and
write operations was simply writing the previous task's benchmarks to a unique
file. This did not appear to affect any other storage device. This appears to imply
that the latency alone of writing to Isilon was quite significant.

\subsection{Added value of persistent memory}

Our results show that Optane has superior performance to that of other storage devices, with the
exception of DRAM. Optane as a persistent memory storage device is expected to bring significant
performance improvements to the processing of neuroimaging data. Input datasets to neuroimaging pipelines
are increasing rapidly in size. Storing such datasets directly on Optane would significantly reduce the
impacts of I/O on the processing. If Optane is located as a persistent memory storage device in HPC clusters,
it could also be used as a burst buffer to network-attached storage devices. This would vastly
improve the performance of standard neuroimaging pipelines, as they produce temporary data files that 
are larger than the input dataset. Writing these temporary files to slow, network-attached storage, can have
significant impacts on the performance of a pipeline.

In the case of high-resolution images, Optane can be leveraged to enable the users to rapidly
extract their regions of interest with minimal I/O costs. Futhermore, Optane would improve the
speed and fluidity of visualization applications on these datasets.

\section{Conclusion}

Optane has been found to drastically reduce the processing time of neuroimaging applications and
can bring the performance close to DRAM speeds. Extending available memory using Optane is also 
expected to help reduce I/O times of write-back devices by potentially extending available cache space.
It is unclear, however, if traditional SSDs can perform just as well as Optane with additional storage devices
attached.

It is believe Optane can be useful for a variety of neuroimaging applications. For
instance, typically applications executing in HPC environments can benefit from 
the speedups provided by Optane. Furthermore, image visualization servers will be able
to display higher resolution images at significantly greater speeds.
\section{Acknowledgement}
We are thankful to Intel and Dell
for the 90-day trial of the Intel Optane DC Server and high-quality technical support.
\bibliographystyle{IEEEtran} 
\bibliography{biblio}

\end{document}
